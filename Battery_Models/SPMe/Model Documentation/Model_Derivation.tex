%\documentclass[12pt,draftcls]{ucdavisthesis}
\documentclass[12pt]{article}

% %\usepackage{bookmark}
% \usepackage[us,nodayofweek,12hr]{datetime}
% \usepackage{graphicx}
% %\usepackage[square,comma,numbers,sort&compress]{natbib}
% %\usepackage{hypernat}
% % Other useful packages to try
\usepackage{amsmath}
\usepackage{amssymb}
\usepackage{amsfonts}
% \usepackage{listings}
% \usepackage{cleveref}
% \usepackage{siunitx}
% \usepackage{subcaption}
% \usepackage{enumerate}
% \usepackage[short]{optidef}
% \usepackage[framed,numbered,autolinebreaks,useliterate]{mcode}
%
%
% \usepackage{epsfig}
% \usepackage{gensymb}
% \usepackage{graphicx}

\title{Single Particle Model with Electrolyte Dynamics Derivation and Implementation}
\author{Jonathan Dorsey}


\begin{document}
\maketitle
\pagebreak

\abstract{In this section I will briefly introduce the ``Single Particle Model'' and it's more accurate exentsion ``Single Particle Model with Electrolyte Dynamics'' as a way of documenting and understanding the basics of this model conceptually, as well as perfroming the requisite mathematics required to concepts into meaningful form which can be directly implemented into a coding lanuage like MATLAB, Python, or C++ for actual simulation purposes. }

\section{Introduction to SPM \& SPMe}
In order to understand the SPM or SPMe models of a Lithium Ion battery, it is first required to understand the basic components of the model which SPM and SPMe are derived from.

\paragraph{} The primary reference for battery modeling comes from a \textit{first principles} mathematical model of battery dynamics called the \textbf{Doyle-Fuller-Newman} model or \underline{DFN} for short.

\paragraph{} The DFN model considers the electrochemical and diffusion dynamics of the battery to construct a \textit{partial differential equations} based model which describes the changes and dynamics of the physical states of the battery would experience while in operation.

\subsection{Doyle-Fuller-Newman Model}
The \textit{Doyle-Fuller-Newman Model} (DFN), is the first principles mathematical expressions which predict the evolution of the solid phase lithium concentration, the electrolytic lithium concentration, the solid electric potential, the electrolyte electric potential, the ionic current, and the molar ion fluxes.

\vspace{.25in}

\[ \frac{\partial{c_s^\pm}}{\partial{t}}(x,r,t) = \frac{1}{r^2}\frac{\partial{}}{\partial{r}}[D_s^\pm r^2\frac{\partial{c_s^\pm}}{\partial{r}}(x,r,t)] \]

\[\epsilon_e^j \frac{\partial{c_e^j}}{\partial{t}}(x,t) = \frac{\partial{}}{\partial{x}}[D_e^eff(c_e^j)\frac{c_e^j}{\partial{x}}(x,t) + \frac{1-t_c^0}{F}i_e^j(x,t)]  \]

for $j \in \{-, sep, + \} $

\[ \sigma^{eff,\pm}\cdot\frac{\partial{\phi_s^\pm}}{\partial{x}}(x,t) = i_e^\pm(x,t) - I(t) \]

\[\kappa^{eff}(c_e) \cdot \frac{\partial{\phi_e}}{\partial{x}} (x,t) = -i_e^\pm(x,t) + \kappa^{eff}(c_e) \cdot \frac{2RT}{F}(1-t_c^0)\times(1+\frac{d\ln f_\c/a}{d\ln c_e}(x,t))\frac{\partial{\ln c_e}}{\partial{x}}(x,t)  \]


\section{First Principles and Fundamental Mathematics}

In this section, I want to cover the fundamental mechanisms and mathematics which describe the operation of batteries on a ``First Principles'' level. These concepts are new to me and bare working through to gain a concrete grasp of the material.

\subsection{Fick's Laws of Diffusion}
The primary mechanism which drives the modeling of batteries is that of \textbf{diffusion}. This mechanism is characterized by atomic/molecular motion or interactions. While this is a gross generalization, it suffices for the study of this material, since I am not knowledgeable about mechanics in the molecular domain. With that being said, diffusion is a dominate means through which heat is transmitted via conduction, and chemicals of varying concentrations interact when put together.

\subsubsection{Fick's First Law of Diffuion}
Fick's First Law relates the spatial changes in \underline{concentration} ($\phi$) to the \underline{diffusive flux} ($\vec{J}$). The differential equation governing this phenomimon is given below.

\vs

In the 1-Dimensional case:

\[ \vec{J} = -D \frac{d\phi}{dx} \]

In the 3-Dimensional case:

\[ \vec{J} = -D \vec{\nabla}\phi \]

Where ``D'' is the constant of diffusion, which is the proportional between the two sides of the equation.

\subsubsection{Fick's Second Law of Diffusion}
Fick's Second Law of Diffusion, uses Fick's First Law as the starting point from which to derive the time dependency of concentration on the Laplacian of the concentration. While there is a reasonable derivation which could be inserted here. It is is sufficient to understand that the time dynamics are merely obtained by extending the Fick's First Law in the differential analysis which was initial used to derive the First Law.

In the 1-Dimensional case:

\[\frac{\partial{\phi}}{\partial{t}} = D \frac{\partial^2{\phi}}{\partial{x^2}}   \]

In the 3-Dimensional case:

\[\frac{\partial{\phi}}{\partial{t}} = D\vec{\nabla}^2\phi \]

where...  \[ \vec{\nabla} = \langle \frac{\partial{}}{\partial{x}}, \frac{\partial{}}{\partial{y}}, \frac{\partial{}}{\partial{z}} \rangle \]

\subsubsection{Coordinate Transformation}

While these equations are in and of themselves very useful in describing the mechanisms of battery operation, they are in the wrong coordinate system. Depending on the battery model being implemented the actual equations for diffusion of lithium are in \textbf{spherical coordinates}. This means that we have one of two choices to make to obtain the diffusion equations in the appropriate form to solve the problem. First is to simply re-derive the equations from using differential analysis techniques natively from spherical coordinates while the second option is transform the differential equations form one coordinate system to another.

Since the first method is often ``very'' straightforward to derive in a particular coordinate system, and rather messy to derive in another, I wanted to go through the exercise of transformation the ``Cartesian Coodinate'' for of the Fick's Second Law into the ``Spherical Coordinates,'' as I feel this is the most useful techinque which I can hopefully apply to other problems in the future.

The primary means of transforming a differential equation from coordinate system to another is to know the direct cordinate relationships between the two systems and then to apply the \textbf{chain rule} to convert undesired differential coordinates into desireable ones. The challenge in this endeavour is most in book keeping and accounting for all differential chain rules which occur.

To demonstrate this process, this example will be to convert Laplace's Equation from Cartesian coordinates to cylindrical coordinates. \newline{}

Given Laplace's Equation:

\[ \nabla^2 = \frac{\partial{}^2}{\partial{x^2}} + \frac{\partial{}^2}{\partial{y^2}} + \frac{\partial{}^2}{\partial{z^2}} \]

The relationships from Cartesian to Cylindrical are given by the following equations:

\[ x = r\cos{\phi}\]
\[ y = r\sin{\phi}\]
\[ z = z \]

Such that ...
\[ r = \sqrt{(x^2 + y^2)}\]


To begin the process of converting each Cartesian differential to the Cylindrical differential we perform the \underline{chain rule} as follows the for X, Y, \& Z coordinates in succession.

\[ \frac{\partial{}}{\partial{x}} = \frac{\partial{r}}{\partial{x}}\frac{\partial{}}{\partial{r}} + \frac{\partial{\phi}}{\partial{x}}\frac{\partial{}}{\partial{\phi}} +\frac{\partial{z}}{\partial{x}}\frac{\partial{}}{\partial{z}} \]

\[ \frac{\partial{}}{\partial{y}} = \frac{\partial{r}}{\partial{y}}\frac{\partial{}}{\partial{r}} + \frac{\partial{\phi}}{\partial{y}}\frac{\partial{}}{\partial{\phi}} +\frac{\partial{z}}{\partial{y}}\frac{\partial{}}{\partial{z}} \]

\[ \frac{\partial{}}{\partial{z}} = \frac{\partial{r}}{\partial{z}}\frac{\partial{}}{\partial{r}} + \frac{\partial{\phi}}{\partial{z}}\frac{\partial{}}{\partial{\phi}} +\frac{\partial{z}}{\partial{z}}\frac{\partial{}}{\partial{z}} \]


\pagebreak

The only part of these equations which we can compuate the differentials relating the new coordinates (in the numerator) with the old coordinates (in the denominator), using the original non-differential relationships shown above.


\[ \frac{\partial{r}}{\partial{x}} = \frac{1}{2}(x^2 + y^2)^{-.5} \cdot 2x = \frac{x}{\sqrt{(x^2 + y^2)}} = \frac{x}{r} = \cos{\phi}\]

\[ \frac{\partial{\phi}}{\partial{x}} = \frac{-\frac{y}{x^2}}{1 + \frac{y^2}{x^2}} = \frac{-y}{x^2 + y^2} = \frac{-y}{r^2} = \frac{-\sin{\phi}}{r}\]

\[ \frac{\partial{z}}{\partial{x}} = 0 \]

When simplified these equations reduce to...

\[ \frac{\partial{r}}{\partial{x}} = \cos{\phi}\]

\[ \frac{\partial{\phi}}{\partial{x}} = \frac{-\sin{\phi}}{r}\]

\[ \frac{\partial{z}}{\partial{x}} = 0 \]

Again using the same process of transformation via chain rule...

\[ \frac{\partial{r}}{\partial{y}} = \frac{1}{2}(x^2 + y^2)^{-.5} \cdot 2y = \frac{y}{\sqrt{(x^2 + y^2)}} = \frac{y}{r} = \sin{\phi}\]

\[ \frac{\partial{\phi}}{\partial{y}} = \frac{x}{x^2 + y^2} = \frac{-\sin{\phi}}{r}\]

\[ \frac{\partial{z}}{\partial{x}} = 0 \]


When simplified these equations reduce to...

\[ \frac{\partial{r}}{\partial{y}} = \sin{\phi}\]

\[ \frac{\partial{\phi}}{\partial{y}} = \frac{\cos{\phi}}{r}\]

\[ \frac{\partial{z}}{\partial{y}} = 0 \]

Again, using this exact same process for the remaining variable to be transform...

\[ \frac{\partial{r}}{\partial{z}} = 0\]
\[ \frac{\partial{\phi}}{\partial{z}} = 0\]
\[ \frac{\partial{z}}{\partial{z}} = 1\]

Which are already in the simplified form of the equations.

Therefore we can construct the the partial with respect to X, Y \& Z respectively as follows.

\[ \frac{\partial{}}{\partial{x}} = \cos{\phi}\frac{\partial{}}{\partial{r}} + \frac{-\sin{\phi}}{r} \frac{\partial{}}{\partial{\phi}} + 0\cdot \frac{\partial{}}{\partial{z}}  \]

\[ \frac{\partial{}}{\partial{y}} = \sin{\phi}\frac{\partial{}}{\partial{r}} + \frac{-\cos{\phi}}{r} \frac{\partial{}}{\partial{\phi}} + 0\cdot \frac{\partial{}}{\partial{z}}  \]

\[ \frac{\partial{}}{\partial{z}} = 0\cdot\frac{\partial{}}{\partial{r}} + 0\cdot \frac{\partial{}}{\partial{\phi}} + 1 \cdot \frac{\partial{}}{\partial{z}}  \]

Now that these differential relationships are computed we can apply them to solve the problem of converting \textit{Laplace's Equation} into cylindrical coordinates. Recall that this equation is given by the expression.

\[\nabla^2} = \frac{\partial{}^2}{\partial{x^2}} + \frac{\partial{}^2}{\partial{y^2}} + \frac{\partial{}^2}{\partial{z^2}} = 0  \]

Since we have only computed the first partial derivatives with respect to each Cartesian coordinates, as expressed by the equations above. To solve this problem, we need to square the first differential transfomations so that we can directly substitute them into \textit{Laplace's Equation}.


\[  \left( \frac{\partial{}}{\partial{x}}\right)^2  = \left[\cos{\phi}\frac{\partial{}}{\partial{r}} + \frac{-\sin{\phi}}{r} \frac{\partial{}}{\partial{\phi}}\right]\left[\cos{\phi}\frac{\partial{}}{\partial{r}} + \frac{-\sin{\phi}}{r} \frac{\partial{}}{\partial{\phi}} \right] \]

\[  \left( \frac{\partial{}}{\partial{y}}\right)^2  = \left[\sin{\phi}\frac{\partial{}}{\partial{r}} + \frac{\cos{\phi}}{r} \frac{\partial{}}{\partial{\phi}}\right]\left[\sin{\phi}\frac{\partial{}}{\partial{r}} + \frac{\cos{\phi}}{r} \frac{\partial{}}{\partial{\phi}} \right] \]


\[ \left(\frac{\partial{}}{\partial{z}}\right)^2 = 1 \]

In theory in these expressions can be substituted directly into \textit{Laplace's Equation} (in its' standard form) and the problem is ``theoretically'' solved. However this expanded form is far from being the most simplified and useful expression for this equation represented in the cylindrical coordinate system.


\[ y = mx + b\]















\end{document}
